\documentclass[10pt, spanish]{beamer}
\usepackage[spanish]{babel}

\usetheme[progressbar=frametitle]{metropolis}

\usepackage{booktabs}
\usepackage[scale=2]{ccicons}

\usepackage{pgfplots}
\usepgfplotslibrary{dateplot}

\usepackage{float}
\graphicspath{ {images/} }

\title{Identificaci\'on de patrones y algoritmos de consolidaci\'on en bases de datos de posicionamiento}
\date{\today}
\author{Pilar Barbero Iriarte}
\institute{Universidad de Zaragoza}
\titlegraphic{\hfill\includegraphics[height=1.5cm]{unizar.png}}

\begin{document}

\maketitle

\begin{frame}
  \frametitle{Table of Contents}
  \setbeamertemplate{section in toc}[sections numbered]
  \tableofcontents[hideallsubsections]
\end{frame}

\section{Introducci\'on}

%%Introducci\'on
\begin{frame}[fragile]
\frametitle{Introducci\'on}
Contexto: 
\begin{itemize}
	\item Empresa Zaragozana de telecomunicaciones.
	\item Almacenamiento de posiciones GPS de sujetos.
	\item Capacidad de guardado de posiciones limitada.
\end{itemize}
\end{frame}


\begin{frame}[fragile]
\frametitle{Introducci\'on}
Problemas:
	\begin{itemize}
		\item Exceso de \'estas.
		\item No existe preprocesado antes de la inserci\'on.
		\item No existe postprocesado despu\'es de la inserci\'on.
		\item No todas aportan informaci\'on.
	\end{itemize}
\end{frame}

%%OBJETIVO
\begin{frame}[fragile]
  \frametitle{Introducci\'n}
  \begin{itemize}
  	  \item Eliminar posiciones repetidas.
  	  \item Eliminar posiciones que no aporten informaci\'on.
   \end{itemize}
\end{frame}

\section{An\'alisis de los datos}

\begin{frame}{An\'alisis de los datos}
  \begin{itemize}[<+- | alert@+>]    
\item \textbf{Id}: Identificador num\'erico
\item \textbf{IdServidor}: Identificador num\'erico del servidor que realiza la inserci\'on
\item \textbf{Recurso}: identificador del sujeto que transfiere la posici\'on
\item \textbf{Latitud}: real que representa la latitud GPS
\item \textbf{Longitud}: real que representa la longitud GPS
\item \textbf{Velocidad}: entero que representa la velocidad instant\'anea
\item \textbf{Orientaci}\'on: entero que representa la orientaci\'on respecto al norte en grados
\item \textbf{Cobertura}: booleano que indica si tiene cobertura (n. sat\'elites)
\item \textbf{Error}: error en la toma de posici\'on
  \end{itemize}
\end{frame}

\section{¿C\'omo abordar el problema?}
\begin{frame}{¿C\'omo abordar el problema?}

T\'enicas a analizar:

  \begin{columns}[T,onlytextwidth]
    \column{0.5\textwidth}
    \metroset{block=fill}
    
      \begin{exampleblock}{Consolidaci\'on simple}
        Consolidaci\'on por distancia\\
        Consolidaci\'on por adelgazamiento\\
        Consolidaci\'on por tiempo
      \end{exampleblock}

    \column{0.5\textwidth}

      \metroset{block=fill}
      \begin{exampleblock}{Algoritmos de \textit{clustering}}
        K-means\\
        DBSCAN\\
        DJ-Cl\'uster
      \end{exampleblock}

  \end{columns}


\end{frame}


\subsection{M\'etodos simples}
\begin{frame}

\end{frame}

\subsection{M\'etodos avanzados}
\begin{frame}

\end{frame}


\section{Comparativa}
\begin{frame}

\end{frame}

\section{Conclusiones}
\begin{frame}

\end{frame}

\section{Demostraci\'on}
\begin{frame}

\end{frame}

\section{Preguntas}

\begin{frame}
	\begin{center}
		\href{http://github.com/pbarbero/TFM}{http://github.com/pbarbero/TFM}
	\end{center}
\end{frame}

\end{document}
