\documentclass[a4paper, 12pt]{article}
\usepackage{listings}


%%Color table

\usepackage{xcolor,colortbl}
\newcommand{\mc}[2]{\multicolumn{#1}{c}{#2}}
\definecolor{Gray}{gray}{0.85}
\definecolor{LightCyan}{rgb}{0.88,1,1}

\newcolumntype{a}{>{\columncolor{Gray}}c}
\newcolumntype{b}{>{\columncolor{white}}c}
%%


\begin{document}

\section{Planteamiento del problema}

\section{Datos y estructura de los datos suministrados}

Se nos suministran dos bases de datos correspondientes a dos ciudades brasile\~nas distintas, Salvador de Bah\'ia y R\'io de Janeiro. En cada de una de ellas encontramos posiciones de distintos sujetos estudiados identificados a trav\'es de un c\'odigo. Cada base de datos contiene una tabla llamada \textit{posicionesgps} en la que encontramos un registro por cada posici\'on tomada por cada sujeto entre los d\'ias 2015-02-17 08:00:05 y 2015-03-04 08:18:05. \\

La estructura de los registros es la siguiente:\\

\begin{center}
	\begin{tabular}{| l | l  |}
	\hline
	\rowcolor{LightCyan}
	\hline
  		\multicolumn{2}{|l|}{Par\'ametros} \\
	\hline
	Id & Identificador num\'erico de la posici\'on (clave primaria) \\
	IdServidor & Identificador num\'erico del servidor que realiza la inserci\'on (PK) \\
	Recurso & Nombre del recurso (tetra:1234567) \\
	Latitud & Real que representa la latitud GPS \\
	Longitud & Real que representa la longitud GPS \\
		Velocidad & Entero que representa la velocidad instant\'anea \\
	Orientaci\'on & Entero que representa la orientaci\'on respecto al norte en grados \\
	Cobertura & Booleano que indica si hay cobertura \\
	Error & Booleano que nos indica si ha habido alg\'un error en la toma de la posici\'on \\
	\hline
	\end{tabular}

\end{center}

\smallskip

En base de datos, el tipo de datos guardado es:

\begin{lstlisting}[language=bash]
mysql> explain posicionesgps;
\end{lstlisting}


\begin{center}
	\begin{tabular}{|l|l|l|l|l|}
	\rowcolor{LightCyan}
	\hline
	Field & Type & Null & Key & Default \\
	\hline
	id & bigint(10) & NO &  PRI & 0  \\
	idServidor & int(10) unsigned & NO & PRI & 0 \\
	recurso & varchar(100) & YES & MUL & NULL  \\
	latitud & double & YES & & NULL  \\
	longitud & double & YES & & NULL  \\
	velocidad & tinyint(10) unsigned & YES & & NULL  \\
	orientacion & smallint(10) unsigned & YES  & & NULL  \\
	cobertura & tinyint(10) unsigned & YES  & & NULL \\
	error & tinyint(10) unsigned  & YES & & NULL  \\
	antigua & tinyint(10) unsigned & YES  & & 0  \\
	fecha\_timestamp & timestamp & NO & MUL & CURRENT\_TIMESTAMP \\
	autom\'atico & tiniyint(10) unsigned & NO & MUL & 0  \\
	\hline
	\end{tabular}
\end{center}

\smallskip


Para este estudio se ha trabajado s\'olo con los siguientes datos,\\

\begin{enumerate}
\item Id
\item Recurso
\item Latitud
\item Longitud
\item Velocidad
\item Fecha
\end{enumerate}


\pagebreak
\subsection{An\'alisis de los datos}

\pagebreak
\subsection{Espacio en disco}

Con la cantidad de posiciones suministradas, cu\'anto ocupa cada posici\'on en disco, para hacernos una idea de cu\'antas posiciones ser\'ia posible acumular en funci\'on de la frecuencia de \'estas sobre un espacio en disco finito.

En nuestra base de datos llamada \textbf{R\'io de Janeiro} contamos con \textbf{6928467} posiciones y en \textbf{Salvador de Bah\'ia} contamos con \textbf{4599974} posiciones.

El tama\~no en disco de nuestras bases de datos es,

\begin{lstlisting}[language=bash, basicstyle=\small]
mysql> SELECT table_schema as `Database`, 
	   table_name AS `Table`,  
	   round(((data_length + index_length) / 1024 / 1024), 2) 
	   FROM information_schema.TABLES  
	   ORDER BY (data_length + index_length) DESC;

\end{lstlisting}

\begin{center}

	\begin{tabular}{| l | l | l | l |}
	\hline
	Database & Table & Size in MB & Size in KB \\
	\hline
	rio & posicionesgps & 1205.64 & 120564000 \\
	bahia & posicionesgps & 961.42 & 96142000 \\
	\hline
	\end{tabular}
\end{center}

Lo cual nos da una idea de cu\'anto puede ocupar una toma de posici\'on en disco.\\

El total de posiciones almacenadas en r\'io es de $6928467$ luego podemos estimar el tama\~no de una posici\'on en, \\
$$\frac{120564000}{6928467} = 17.4012519653 KB$$

El total de posiciones almacenadas en bah\'ia es de $4599974$, luego\\
$$\frac{96142000}{4599974} = 20.9005529162 KB$$

Podemos aproximar el tama\~no de una posici\'on por unos $19$ KB. \\

Supongamos que una consola tiene unos $1$GB de almacenamiento. Podemos almacenar unas $52631$ posiciones en estos 30GB. 

Los datos han sido recogidos entre las fechas 2015-02-17 08:00:05 y 2015-03-04 08:18:05, lo que hace una diferencia de 360 horas.

Tenemos 5014 distintos tipos de sujetos a estudiar en la base de datos de r\'io:

\begin{lstlisting}
mysql> USE rio;
mysql> SELECT COUNT(distinct(recurso)) 
	   FROM posicionesgps;
\end{lstlisting}

\begin{center}
 \begin{tabular}{|l|}
 \hline
	count(distinct(recurso)) \\
 \hline
	5014 \\
 \hline
 
 \end{tabular}
\end{center}


Lo que nos da una frecuencia de toma de :

$$ \frac{6928467}{5014 \cdot 360} = 3.83$$ posiciones a la hora.

Si aument\'aramos esta frecuencia a una posici\'on cada 30 segundos, conseguri\r'iamos una frecuencia de 120 posiciones a la hora, luego un \'unico sujeto, en una jornada laboral de 8 horas, ocupar\'ia en espacio de 19.2 MB. 


\pagebreak
\section{Nociones de distancia y vecindario}

Con el fin de realizar los algoritmos de consolidaci\'on, hemos realizado un estudio acerca de distintos tipos de vecindarios a utilizar para los algoritmos de consolidaci\'on propios y los algoritmos de \textit{clustering} utilizados en \textsc{Weka}.

\subsection{Vecindario simple}

Utilizando la distancia eucl\'idea, definimos un vecindario como aquel conjunto de puntos que se  encuentran a una distancia eucl\'idea menor que $\epsilon$ con respecto su centro $p_0$, es decir,

$$ d_E(p_0, p) = \sqrt{lat_{p} - lat_{p_0})^2 + (long_{p} - long_{p_0})^2 } < \epsilon $$

donde $p$ es un punto con latitud $lat_{p}$ y longitud $long_{p}$.

\subsection{Vecindario involucrando el m\'odulo de la velocidad}

En el momento que se toma la posici\'on $p_0$, aparte de la latidud y su longitud, se toma la velocidad instant\'anea del sujeto. Podemos considerar en este caso que, dado que nuestro sujeto se  encuentra a mayor velocidad, puntos m\'as alejados de lo que considerar\'iamos en el primer caso (fuera de nuestro vecindario simple), podr\'ian estar dentro de nuestro nuevo radio, que depender\'ia de la velocidad instant\'anea. As\'i, definimos nuestro nuevo vecindario:

$$ d_E(p_0, p) = \sqrt{lat_{p} - lat_{p_0})^2 + (long_{p} - long_{p_0})^2 } < \epsilon \cdot vel_{p_0} $$

donde $vel_{p_0}$ es la velocidad instant\'anea de nuestro punto centro.

\subsection{Vecindario involucrando el m\'odulo de la velocidad y la orientaci\'on}

Igual que contamos con la velocidad instant\'anea del sujeto, contamos tambi\'en con el dato de la orientaci\'on respecto al norte de nuestro sujeto muestreado. Esta medida est\'a tomada en grados sexagesimales en el sentido de las agujas del reloj respecto al norte.

Gracias a este dato, podemos calcular el vector direcci\'on que contiene la informaci\'on de la orientaci\'on de nuestro sujeto y como tambi\'en conocemos el m\'odulo de la velocidad, obtener el vector velocidad. 

Nuestra componente $x$ que identificaremos con el eje del vector direcci\'on ser\'a el coseno de nuestra orientaci\'on,

$$ cos(or_{p_0})$$

Y nuestra componente $y$ del vector direcci\'on ser\'a el seno de nuestra orientaci\'on,

$$ sin(or_{p_0}) $$

\subsection{Vecindad t0-alcanzable}

Si fijamos un intervalo de tiempo $t_0$, podemos definir una vecindad $t_0$-alcanzable como aquellos puntos que nuestro sujeto puede alcanzar en un tiempo $t_0$. Un sujeto que se desplace a velocidad reducida, tendr\'a una vecindad $t_0$-alcanzable m\'as reducido  que otro que se desplace a una velo$vel_{p_0}\cdot t_0$. 

$$ d_E(p_0, p) = \sqrt{lat_{p} - lat_{p_0})^2 + (long_{p} - long_{p_0})^2 } < vel_{p_0} \cdot t_0 $$

\'Este es un caso concreto del vecindario involucrando la velocidad. 

\subsection{Vecindario involucrando el tiempo}

Las posiciones de nuestros sujetos vienen muestreadas adem\'as con el instante en el que fueron tomadas. Podemos considerar que el tiempo entre tomas tambi\'en es una distancia y definir un vecindario. Definimos esta distancia temporal como la resta de ambos instantes, y el vecindario como:

$$ d_T(p_0, p) = time_p - time_{p_0} < \delta $$

\pagebreak
\section{Algoritmos de consolidaci\'on simples}

\section{K-means}
\section{DJ-Cluser}
\section{Canocopy}

\section{Conclusiones}



\end{document}