\documentclass[a4paper, 12pt]{article}
\usepackage{listings}


%%%%%%%%%%%%%%%%%%%%PYTHON CODING

% Default fixed font does not support bold face
\DeclareFixedFont{\ttb}{T1}{txtt}{bx}{n}{11} % for bold
\DeclareFixedFont{\ttm}{T1}{txtt}{m}{n}{11}  % for normal

% Custom colors
\usepackage{color}
\definecolor{deepblue}{rgb}{0,0,0.5}
\definecolor{deepred}{rgb}{0.6,0,0}
\definecolor{deepgreen}{rgb}{0,0.5,0}

\usepackage{listings}

% Python style for highlighting
\newcommand\pythonstyle{\lstset{
language=Python,
basicstyle=\ttm,
otherkeywords={self},             % Add keywords here
keywordstyle=\ttb\color{deepblue},
emph={MyClass,__init__},          % Custom highlighting
emphstyle=\ttb\color{deepred},    % Custom highlighting style
stringstyle=\color{deepgreen},
frame=tb,                         % Any extra options here
showstringspaces=false,           % 
basicstyle=\small,
}}


% Python environment
\lstnewenvironment{python}[1][]
{
\pythonstyle
\lstset{#1}
}
{}

% Python for external files
\newcommand\pythonexternal[2][]{{
\pythonstyle
\lstinputlisting[#1]{#2}}}

% Python for inline
\newcommand\pythoninline[1]{{\pythonstyle\lstinline!#1!}}
%%%%%%%%%%%%%%%%%%%%%%%%%%%%%%%%%


\begin{document}


\section{Definici\'on de posici\'on}

De las posiciones suministradas, obtenemos s\'olo los datos que nos interesan y creamos una clase en Python con las siguientes propiedades:

\begin{python}
class Position:
    def __init__(self, id, resource, lat, lon, speed, track, date):
    		self.id = id
        self.resource = resource
        self.lat = lat
        self.lon = lon
        self.speed = speed
        self.track = track
        self.date = date
\end{python}

Definimos si un punto est\'a dentro del vecindario creado por otro seg\'un distintos tipos de distancias

Primero de todo, definimos la distancia eucl\'idea:

\begin{python}
	def distance_eu(self, q):
    		return math.sqrt((self.lat - q.lat)**2 + (self.lon - q.lon)**2)
\end{python}

De la cual definimos el m\'etodo con respecto a esta de si un punto se encuentra en el vecindario de otro:

\begin{python}
	def is_in_neighborhoodByEUSimple(self, q, eps):
    		return self.distance_eu(q) < eps
\end{python}

Definimos la noci\'on de vecindario en funci\'on de la velocidad que tiene el sujeto en un instante dado y la distancia que podr\'ia alcanzar en un periodo $t_0$, es decir,

\begin{python}
	def is_in_neighborhoodT0Reachable(self, q, t0):
    		return self.distance_eu(q) < t0 * self.speed
\end{python}

Y definimos tambi\'en la noci\'on de vecindario pero esta vez utilizando el dato de la fecha,

\begin{python}
	def is_neighboorhoudByTime(self, q, lapse):
    		foo = time.mktime(self.date.timetuple())
        bar = time.mktime(q.date.timetuple())
		return abs(foo - bar) < lapse

\end{python}


\section{Cl\'uster}

Definimos cl\'uster como:

\begin{python}
class Cluster:
    def __init__(self, center, points):
		self.center = center
        self.points = points
\end{python}

Es interesante definir la propiedad de un si un punto pertenece a un cl\'uster como m\'etodo propio de la clase de Position:

\begin{python}
def isInCluster(self, cluster):
	for q in cluster.points:
		if self.id == q.id:
			return True

	return False
\end{python}

Para el algoritmo de Dj-Cluster, es interesante definir la propiedad de Density joinable,

\begin{python}
def isDensityJoinable(self, listClusters):
	for cluster in listClusters:
    		for p in self.points:
        		if p.isInCluster(cluster):
            		return cluster

	return None
\end{python}

Y la noci\'on de mergear dos clusters,

\begin{python}
def mergeCluster(self, cluster):
	for p in cluster.points:
    		if not p.isInCluster(cluster):
        		self.points.append(p)

    return self
\end{python}


\end{document}